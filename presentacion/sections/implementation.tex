\section{Implementación}
\subsection{¿Cómo se implementó el Moogle?}
\begin{frame}
    ¿Cómo se implementó el Moogle?
    \vspace{0.5cm}
    {
        \begin{columns}
            \begin{column}{0.3\textwidth}
                ModVec
            \end{column}
            \begin{column}{0.8\textwidth}
                La clase donde se realiza todo el procesamiento de
                los textos de los documentos para realizar los
                cálculos del Modelo Vectorial utilizando las
                formulas del TF-IDF,  los cálculos para calcular el
                peso de cada palabra de la búsqueda del usuario, y
                la similitud de coseno para calcular el score de
                cada documento según su relevancia.

            \end{column}
        \end{columns}
    }
    \pause
    \vspace{0.5cm}
    {
        \begin{columns}
            \begin{column}{0.3\textwidth}
                Inicio
            \end{column}
            \begin{column}{0.8\textwidth}
                La clase donde se inicializan los datos,
                sobre los documentos, que se van a utilizar
                para cada búsqueda. Datos que se pueden
                crear antes de que el usuario haga la
                búsqueda. \\
                Ej. Diccionario IDF$<$palabras,IDF de la palabra$>$


            \end{column}
        \end{columns}
    }
\end{frame}
\begin{frame}
    ¿Cómo se implementó el Moogle?
    \vspace{1cm}
    {
        \begin{columns}
            \begin{column}{0.3\textwidth}
                Búsqueda
            \end{column}
            \begin{column}{0.8\textwidth}
                La clase donde se trabaja con la query
                para crear una sugerencia, y para despues
                con el método ModVec calcular el peso de
                cada una de sus palabras.

            \end{column}
        \end{columns}
    }
    \pause
    \vspace{1cm}
    {
        \begin{columns}
            \begin{column}{0.3\textwidth}
                Ops
            \end{column}
            \begin{column}{0.8\textwidth}
                La clase con la que después de calcular el
                score de cada documento se le aplica
                el efecto de los distintos operadores
                ! , $\wedge$ , $\sim$ , *  a los resultados


            \end{column}
        \end{columns}
    }
\end{frame}

\subsection{Funcionamiento}
\begin{frame}
    \centering
    {\large ¿Cómo funciona Moogle!? \par}
    %\vspace{.5cm}
    \begin{enumerate}
        \item Cuando se ejecuta el proyecto,
              se ponen en funcionamiento una serie de métodos
              que procesan el texto de los documentos,
              los métodos de la clase Inicio, que llaman a
              algunos métodos de la clase ModVec, y para cuando
              aparece la pagina web del Moogle está creado un
              Diccionario de

              $<$documentos, Diccionario$<$ palabras del documento, TF-IDF de cada palabra$>>$

    \end{enumerate}



\end{frame}
\begin{frame}
    \centering
    {\large ¿Cómo funciona Moogle!? \par}
    %\vspace{.5cm}
    \begin{enumerate}
        \item Cuando introducimos una busqueda en
              nuestra página, se comienza a ejecutar el
              método Query de la clase Moogle, que:

              \begin{enumerate}[\alph{enumii}]
                  \item Primero trabaja con la query hasta crear
                        la sugerencia, utilizando los métodos de la
                        clase Busqueda, y 	después, utilizando los
                        métodos de la clase ModVec crea un
                        Diccionario de

                        $<$palabras de la query, peso de cada palabra$>$
                  \item Posteriormente utilizando la clase ModVec,
                        y los archivos creados al iniciar el Proyecto,
                        se calcula la similitude de coseno de cada
                        documento con la query, asignándole un score a
                        cada document y creando un Diccionario de

                        $<$Documentos, Score del Documento$>$

                  \item Despues utilizando los metodos de la clase
                        Ops, se le aplica el efecto de los distintos
                        operadores a ese 	diccionario, quedándonos
                        con una parte de los documentos que nos
                        importan para dar como resultados

              \end{enumerate}
    \end{enumerate}



\end{frame}

\begin{frame}
    \begin{enumerate}[\alph{enumi}]

        \item Y a ese Diccionario en dependencia de si
              todas las palabras de la búsqueda tienen el
              operador !  o no se 	eliminan los documentos
              con score 0 o no

        \item Después se ordenan los documentos y se
              da como resultado un máximo de 5 documentos,
              y además la sugerencia, que es lo que devuelve
              el método Query.

                  {\large Y esos documentos y la sugerencia que
                      son devueltas son mostradas en la pagina
                      donde el usuario hizo la búsqueda}


    \end{enumerate}

\end{frame}


