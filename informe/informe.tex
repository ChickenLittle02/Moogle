\documentclass[12pt, a4paper]{article}
%pt define el tamaño de la fuente, y el parametro de la derecha
% define el tamaño de la hoja

%\usepackage[spanish]{babel}
%COn este paquete todos los subtítulos definidos por Latex en vez de salir en Inglés salen en español
%Por ejemplo en abstract, con este paquete en vez de poner abstract pone Resumen
%SI lo quitamos pone abstract de nuevo

\title{Informe de Proyecto de Progamación\\ “Moogle”}
\author{Rubén Martínez Rojas}
\date{}
\begin{document}
\begin{titlepage}
    \maketitle
\end{titlepage}

\begin{abstract}
    Moogle es una página web creada con el fin de buscar
    un texto en un grupo de archivos. Para su funcionamiento
    utiliza un Sistema de Recuperación de Información
    desarrollado en lenguaje C\#
\end{abstract}

\section{Introducción}

Para la creación del Proyecto se implementaron diferentes clases en MoogleEngine:
\begin{itemize}
    \item ModVec: La clase donde se realiza todo el procesamiento de los textos de los documentos
          para realizar los cálculos del Modelo Vectorial utilizando las fórmulas del TF-IDF,
          los cálculos para obtener el peso de cada palabra de la búsqueda del usuario,
          y la similitud de coseno para calcular el score de cada documento según su relevancia.
    \item Inicio: La clase donde se inicializan los datos, y se completan con la información sobre
          los documentos que se van a utilizar para cada búsqueda. Datos que se pueden crear
          antes de que el usuario haga la búsqueda.
          Ej. Diccionario IDF de <string,double> con <palabras , IDF de la palabra>
    \item Búsqueda: La clase donde se trabaja con la query para crear una sugerencia,
          y para después con el método ModVec calcular el peso de cada una de sus palabras
          %Ops: La clase con la que después de calcular el score de cada documento se le aplica el efecto de los distintos operadores \! \^ \~ \* a los resultados

          %Ver por qué esta última línea me está dando bateo y no me deja ponerlo como un item
\end{itemize}
\section{Funcionamiento}

Cómo funciona Moogle! ??

\begin{enumerate}
    \item Cuando se ejecuta el proyecto, se ponen en funcionamiento una serie de métodos que procesan el texto de los documentos, son los métodos de la clase Inicio, que llaman a algunos métodos de la clase ModVec, y para cuando aparece la página web del Moogle está creado un Diccionario de
          <documentos, Diccionario< palabras del documento, TF-IDF de cada palabra>
    \item Cuando introducimos una búsqueda en nuestra página, se comienza a ejecutar el método Query de la clase Moogle, que:
          \begin{enumerate}
              \item Primero trabaja con la query hasta crear la sugerencia, utilizando los métodos de la clase
                    Búsqueda, y después, utilizando los métodos de la clase ModVec crea un Diccionario de
                    <palabras de la query, peso de cada palabra>
              \item Posteriormente utilizando la clase ModVec, y los archivos creados al iniciar el Proyecto, se calcula la similitud de coseno de cada documento con la query, asignándole un score a cada documento y creando un Diccionario de
                    <Documentos, Score del Documento>
              \item Después utilizando los métodos de la clase Ops, se le aplica el efecto de los distintos operadores a ese diccionario, quedándonos con una parte de los documentos que nos importan para dar como resultados
              \item Y a ese Diccionario en dependencia de si todas las palabras de la búsqueda tienen el operador ! o no, se le eliminan los documentos con score 0.
              \item Después se ordenan los documentos y se dan como resultado los 5 primeros documentos, y además la sugerencia, que es lo que devuelve el método Query.
                    Y esos documentos y la sugerencia que son devueltas son mostradas en la página donde el usuario hizo la búsqueda
                    %Ver como cambiar para que aquí salga con números romanos en vez de con letras        
          \end{enumerate}

\end{enumerate}

\section{Ejecución del proyecto}

Cuando el proyecto comienza a ejecutarse antes de abrirse la página web
en la que el usuario introduce la búsqueda, se ejecuta el método Main
de la clase Inicio de MoogleEngine, y ahí se comienza a hacer lo siguiente:

\begin{enumerate}
    \item Primeramente, hay que acceder a la carpeta donde están todos los documentos
          para comenzar a procesar los textos de cada uno; para acceder a ella
          utilizamos en la clase Directory el método CurrentDirectory, que nos
          devuelve el path de acceso a la carpeta donde se está corriendo el método,
          ese path lo modificamos con el método ToPath de la clase ModVec y ahí ya
          tenemos el path de acceso a la carpeta donde se almacenan todos
          los documentos.
    \item Con el path de acceso a la carpeta donde están todos los documentos
          se utiliza de la clase Directory el método GetFiles que devuelve
          un array de string con los path a cada archivo txt que este guardado
          en la carpeta Content. Después a partir de esos path creamos un
          Diccionario de tipo <string,string> donde guardamos
          <nombre de documento, texto del documento>, aquí el texto del documento
          se guarda con las letras en minúsculas pues de esta manera es más fácil
          contar las palabras, pues sino Papa y papa serian palabras distintas,
          todo esto lo hacemos utilizando el método DocText de la clase ModVec.
          Después recorriendo ese Diccionario, creamos otro nuevo Diccionario de
          <string, string[ ]> donde guardamos
          <nombre del documento, palabras separadas del documento>,
          estas palabras ya están sin signos de puntuación ni espacios en blanco,
          y esto lo hacemos utilizando el método ToTextDivide de la clase ModVec.
    \item Con el método ToTF de la clase ModVec, creamos un Diccionario de
          <string,Diccionario<string, double>> donde guardamos
          <nombre del documento,<palabras del documento, TF de cada palabra en el documento>.\\

          La fórmula que usamos para calcular el TF de una palabra
          en un documento = freq/MaxFreq;\\
          Freq = las veces que aparece la palabra en el documento;\\
          MaxFreq = la frecuencia de la palabra que más veces aparece en el documento;\\
          Con el método ToTF vamos recorriendo cada documento del diccionario anterior,
          por cada documento creamos un diccionario, e iteramos por el array de
          palabras del documento anterior, si la palabra en ese diccionario no
          está la agregamos con double 1 que sería la frecuencia, aparece una vez,
          si ya está le sumamos uno al double, la frecuencia,
          y guardamos el valor máximo de frecuencia, después que tengamos
          la frecuencia de cada palabra, volvemos a iterar por ese diccionario y
          dividimos cada frecuencia por el valor máximo de frecuencia en ese documento,
          y así tenemos el TF de cada palabra. Cuando una palabra aparezca mucho va a ser
          relevante para ese documento por tanto su valor de TF va a ser mayor.

    \item Para calcular el IDF usamos la fórmula log(Freq/Length)\\
          Freq = Cantidad de documentos en los que aparece esa palabra\\
          Length = Total de documentos\\
          Cuando una palabra aparezca en muchos documentos su valor de IDF va a estar más cerca de 1 y Log(1) va a estar más cerca de cero entonces, y su valor de IDF va a ser menor mientras aparezca en mas documentos
          El IDF lo calculamos con el método ToIDF de la clase ModVec, que recibe el Diccionario <nombre del documento,<palabras del documento, TF de cada palabra en el documento> donde aparecen repetidas por documento la palabra una sola vez, y a partir de él crea un nuevo Diccionario de <string,double> compuesto por <palabras de todos los documentos, IDF de cada palabra>, primeramente va guardando la cantidad de documentos en que aparece la palabra, recorriendo cada documento del diccionario TF y y verificando si tiene ya agregada la palabra al Diccionario IDF, en caso de tenerla le suma 1 a su frecuencia de aparición; y en caso de no tenerla agrega la palabra, y le pone como frecuencia 1. Después vuelve a iterar por cada palabra, dividiendo cada frecuencia por el total de documentos y calculando el logaritmo de esa división, dejando creado un diccionario con los valores TF de cada palabra.
          Ya teniendo ambos diccionarios TF e IDF, con el método ToTF\_IDF de la clase ModVec,
          calculamos el TF\_IDF que no es más que el producto del valor TF de cada palabra
          en el documento, por el valor IDF de la palabra. Este método recibe ambos diccionarios y
          devuelve un nuevo Diccionario de <string,<string,double>> que contiene
          <nombre del documento,<palabra, TF\_IDF de la palabra en ese documento>>,
          y lo que hace es ir iterando por cada documento del diccionario TF, y por cada documento
          va iterando por todas las palabras del IDF, verificando si están en el documento,
          en caso de estarlo multiplican el valor TF*IDF de esa palabra y ya guardan en ese
          valor como TF\_IDF, en caso de no estarlo es porque la palabra no aparece en el documento,
          por tanto su TF es 0, y al multiplicarlo anularía todo el resultado, por tanto se
          pone directamente que el valor TF\_IDF de esa palabra en ese documento es 0.
          Aquí se termina de ejecutar el método Main, y posteriormente aparece la página del Moogle
          donde el usuario puede hacer sus búsquedas, al hacer una búsqueda se comienza a ejecutar
          el método Query de la clase Moogle, que debe dar como resultado los documentos a mostrar y
          las sugerencias.

    \item Con el diccionario de TF\_IDF solo nos quedaría trabajar con la query para hacer la
          sugerencia, crearle un peso a cada palabra de la query, y con la similitud de
          coseno hacer un sistema de scores que represente la relevancia de
          cada documento con respecto a la query.//
          \begin{itemize}
              \item  Primeramente para hacer la sugerencia, trabajamos con la query que
                    se recibe en un string.
              \item con el método DivideQuery de la clase búsqueda,
                    convertimos todas las palabras de la query a minúsculas, y guardamos en un array
                    la query con todas las palabras divididas.
              \item Después con el método Clean de la clase Busqueda,
                    creamos un array de la query nuevo, con las palabras divididas y
                    sin los signos de puntuación, pero con los operadores.
              \item Después con el método LowerString de la clase Busqueda,
                    creamos un array nuevo con las palabras del array anterior pero
                    con todas las palabras sin los operadores.
              \item Después con el método SearchTheOne de la clase Busqueda
                    donde creamos un nuevo array con las palabras de la query
                    como el creado anteriormente, vamos verificando

                    \begin{itemize}
                        \item si las palabras del array de palabras existen en el diccionario IDF
                              que tiene todas las palabras, en caso de hacerlo las mantenemos igual,
                              y en caso de no existir
                        \item verificamos también que esa palabra no sea “”
                              porque en ese caso quiere decir que era o un signo de puntuación o un operador,
                              y en ese caso no seria una palabra, por tanto no habría que buscarle
                              ninguna palabra parecida,
                        \item En caso de no existir, y que no sea “”,
                              con la distancia levenshtein calculamos la palabra más parecida a ella entre todas
                              las del IDF, guardando siempre la que menor distancia tenga y con esa
                              nos quedamos, sustituyéndola entonces en la query.


                    \end{itemize}
              \item Después con el método Change de la clase Busqueda,
                    vamos iterando por cada palabra del array creado por SearchTheOne si es diferente
                    a la palabra creada por el método LowerSTring, significa que la palabra
                    cambio al aplicar la Levenshtein, en ese caso con el método Replace de la clase
                    string, vamos a la posición del array con las palabras en minúscula y
                    con signos de puntuación, y cambiamos la palabra vieja por la nueva, y al final
                    con Split volvemos a unir todo el array de palabras ya modificado y esa es la
                    sugerencia\\
                    Después con el método ToTFQuery de la clase ModVec calculamos el
                    tf de la palabra, con la misma fórmula vista anteriormente freq/MaxFreq,
                    y lo hacemos a partir del array de palabras de la query que no ha
                    sido modificado por la levenshtein, contamos las veces que aparece
                    cada palabra, buscamos la frecuencia mayor y las dividimos,
                    y guardamos un diccionario de<string, double>,
                    con <palabras, tf de cada palabra>\\
                    Posteriormente con el método ToQueryTF\_IDF de la clase ModVec
                    creamos un diccionario de <string, double> que contiene
                    <palabras de la query, peso de la palabra> y se calcula su el peso de
                    cada palabra con la fórmula
                    (a+(1-a)*TF de la palabra)*IDF de la palabra , Donde a = 0,5;\\
                    Entonces lo que hace este método es ir iterando por el diccionario de
                    <palabras, idf de la palabra> creado anteriormente para los documentos ,
                    si la palabra aparece en el diccionario de <palabras de la query, tf de la palabra>,
                    aplicamos la formula anterior y la agregamos con ese valor de peso, en caso de no aparecer
                    en el diccionario de palabras de la query le ponemos como valor 0, de esta manera
                    si hay alguna palabra en la query que no aparezca en ningún documento es obviada.\\
                    Después se ejecuta el método SearchScore de la clase ModVec,
                    que crea un diccionario de <Documentos, peso del documento> \\
                    Para el Score utilizamos la fórmula de similitud de coseno, que busca la similitud
                    entre el vector de palabras de cada documento, y el vector de palabras de la query,
                    con la fórmula\\
                    Numerador = Sumatoria del producto del peso de la palabra en la query por el peso de
                    la palabra en el documento\\
                    Denominador = Raíz cuadrada de Sumatoria A por Raíz cuadrada de Sumatoria B.\\
                    Sumatoria A = Sumatoria de los cuadrados de los pesos de cada palabra de la query\\
                    Sumatoria B = Sumatoria de los cuadrados de los pesos de cada palabra del documento.\\
                    El método recibe el diccionario de <documentos, <palabra del documento, tf\_idf de la palabra>>\\
                    Y el diccionario de <palabras de la query, peso de la palabra>\\
                    Por cada palabra de la query en el diccionario de palabras en la query,
                    va calculando la sumatoria de los pesos de la palabra en la query,
                    y va calculando la sumatoria de los pesos de la palabra en el documento, y
                    va calculando la sumatoria del producto del peso de la palabra en la query por
                    el peso de la palabra en el documento, al final le halla la raíz cuadrada a
                    la sumatoria A y a la sumatoria B, si el producto de las dos raíces es 0,
                    el valor del score es 0 automáticamente, sino se hace la división y
                    ese es el valor del score.


              \item Posteriormente con el método cleanDocs de la clase Ops,
                    se aplica el efecto de cada operador a los resultados. Los operadores son:\\
                    ! si aparece delante de alguna palabra significa que la palabra
                    no puede existir en el documento\\
                    \^ si aparece delante de alguna palabra significa que la palabra tiene
                    que existir obligatoriamente en el documento\\
                    \* Si aparece delante de una palabra, quiere decir que si la palabra está
                    en alguno de los resultados, ese resultado tiene mayor relevancia,
                    mientras más veces se repite el operador más relevancia tiene\\
                    \~ Aparece entre 2 palabras, quiere decir que mientras más cerca estén esas
                    dos palabras más relevancia tiene el documento\\
                    En caso de aparecer más de un operador delante de una palabra,
                    solo se tendría en cuenta el primero de ellos.\\
                    Entonces lo que va haciendo este método es, por cada operador va reccorriendo
                    el array de palabras con operadores que habíamos creado anteriormente, \\

                    si la palabra tiene el operador !\\
                        - Verifica en el diccionario de
                              <documentos, diccionario <palabras del documento, tf de la palabra>>,
                              si el diccionario en ese documento contiene la palabra,
                              se elimina el documento del diccionario de score.\\

                    SI la palabra empieza con el operador \^\\
                    - Verifica en el diccionario de
                              <documentos, diccionario <palabras del documento, tf de la palabra>>,
                              si el diccionario en ese documento no contiene la palabra,
                              se elimina el documento del diccionario de score\\

                    SI la palabra empieza con el operador \*\\
                    - Se cuenta cuantas veces aparece el operador\\
                    - Verifica en el diccionario de
                              <documentos, diccionario <palabras del documento, tf de la palabra>>,
                              si el diccionario en ese documento contiene la palabra,
                              se multiplica el score de ese documento por la cantidad de veces que
                              se repite el operador + 1.\\


                    SI la palabra comienza con el operador \~\\
                    - Primero se verifica que no aparezca en la primera palabra,
                              si aparece en esa palabra no se aplica\\
                    - SI aparece en cualquier otra posición,
                              se verifica en el diccionario de
                              <documentos, diccionario <palabras del documento, tf de la palabra>>,
                              si el diccionario en ese documento contiene la palabra y la palabra anterior,
                              en ese caso se cuenta cuantas palabras hay de diferencia entre las dos,\\
                              o si la distancia es 0, quiere decir que es la misma palabra, y se le suma 1 al score
                              de ese documento.\\
                              o si la distancia es distinta de 0, se divide 1/distancia y el resultado se le suma
                              al score de ese documento\\


          \end{itemize}

          Después con el método SortScores de la clase ModVec se ordena el diccionario 
          resultante, guardándolo en una lista ordenada.\\
          Y Después se dan los resultados:\\
          - En el caso de que todas las palabras de la query empiecen con el operador ! 
          significa que ninguna de esas palabras pueden aparecer, 
          por tanto se quedarían solamente los documentos con score 0, 
          si la lista está vacía, se devuelve SearchItem con 
          (“”,”No hay resultados para su busqueda”,0) y 
          la sugerencia elaborada anteriormente, y eso es lo que se muestra como resultado.\\
          Si la lista tiene menos de 5 elementos, se devuelve un SearchItem con 
          <nombre del documento, snippet del documento, score del documento> 
          de cada documento, y la sugerencia.\\
          Y si tiene más de 5 elementos, se muestran los 5 primeros documentos\\
          En el caso de que todas las palabras de la query no empiecen con el operador !
          se eliminan los documentos con score 0, y se repiten las mismas condiciones 
          para los resultados dichas anteriormente, en dependencia de la cantidad de 
          documentos que queden en la lista.\\
          \\
          También hay una clase Matrices en la que:\\
          
          EL método Suma, recibe dos matrices A y B, donde A tiene que tener 
          la misma cantidad de filas y de columnas que tiene B, A y B son matrices de MxN 
          y el método devuelve la suma de A y B\\
          EL método Resta, recibe dos matrices A y B, donde A tiene que tener la 
          misma cantidad de filas y de columnas que tiene B, A y B son matrices de MxN y 
          el método devuelve la resta de A y B\\
          EL método Producto, recibe dos matrices A y B, donde A tiene que tener la misma 
          cantidad de columnas que filas tiene B, A es una matriz de MxN y B es una matriz 
          de NxK y el método devuelve el producto de A y B.\\
          Para el producto de una matriz por un vector, se puede utilizar este mismo método, 
          donde A sería una matriz de MxN y el vector tendría que cumplir ser una matriz 
          de NX1\\
          En caso de no cumplirse las condiciones necesarias para realizar la suma, resta y 
          la multiplicación, cada método devolverá una matriz nula de 1x1\\
          También está el método ProductoWithScalar, 
          que recibe un escalar y una matriz, y realiza el producto de un escalar por una 
          matriz.\\







\end{enumerate}




\end{document}







